%%%%%%%%%%%%%%%%%%%%%%%%%%%%%%%%%%%%%%%%%
%
% Edited by: J. Harper Cline, August 20th, 2021
% Copying and distribution of this file, with or without modification,
% are permitted in any medium without royalty provided the copyright
% notice and this notice are preserved. This file is offered as-is,
% without any warranty.
%
%----------------------------------------------------------------------------------------
%	Forked From
%----------------------------------------------------------------------------------------
% Medium Length Professional CV
% LaTeX Template
% Version 2.0 (8/5/13)
%
% This template has been downloaded from:
% http://www.LaTeXTemplates.com
%
% Original author:
% Trey Hunner (http://www.treyhunner.com/)
%
% Important note:
% This template requires the resume.cls file to be in the same directory as the
% .tex file. The resume.cls file provides the resume style used for structuring the
% document.
%
%%%%%%%%%%%%%%%%%%%%%%%%%%%%%%%%%%%%%%%%%
\documentclass{resume}
\usepackage[left=0.4 in,top=0.4in,right=0.4 in,bottom=0.4in]{geometry}
\usepackage{hyperref}
\newcommand{\tab}[1]{\hspace{.2667\textwidth}\rlap{#1}} 
\newcommand{\itab}[1]{\hspace{0em}\rlap{#1}}
\name{James Harper Cline}
\address{1511 Wild Azalea Ln. Athens, GA 30606 \\ (706)-799-6061 \\ harper.cline@gatech.edu}
\address{United States Citizen \\  Clearance Level: Secret}
\begin{document}
\vspace{-1em}
%----------------------------------------------------------------------------------------
%	EDUCATION SECTION
%----------------------------------------------------------------------------------------
\begin{rSection}{EDUCATION}
	{\bf Bachelor of Science in Computer Engineering} \hfill {August 2017 - Present}
	\\ 
	Georgia Institute of Technology, Atlanta, GA. GPA: 3.38
\end{rSection}
%----------------------------------------------------------------------------------------
%	TECHNICAL STRENGTHS SECTION
%----------------------------------------------------------------------------------------
\begin{rSection}{SKILLS}
	\begin{tabular}{ @{} >{\bfseries}l @{\hspace{3ex}} l }
	Programming & Proficient: C\#, C++, C, Python \\
	 & Beginner: Java, MATLAB, MIPS Assembly, SystemVerilog, LaTeX \\ 
	Platforms &  Ubuntu Server, Ubuntu Desktop, Red Hat Enterprise, Jenkins, Kubernetes  \\
	Hardware & ARM Mbed, Arduino Uno, TERMA ALQ-213, Altera DE2 FPGA \\
	Software & Visual Studio (Windows Presentation Foundation), Atlassian Suite, Autodesk EAGLE, JetBrains \\
	 & Software \\
	\end{tabular}
\end{rSection}
%----------------------------------------------------------------------------------------
%	EXPERIENCE SECTION
%----------------------------------------------------------------------------------------
\begin{rSection}{EXPERIENCE}
	\begin{rSubsection}{Software Engineering Co-Op at GTRI}{May~2020 - December 2020}{}{}{Atlanta, GA}
		\item Refactored the previous Event Aggregator Windows Presentation Foundation (WPF) component, improving \\ maintainability and portability
		\item Developed an Event Aggregator test suite to use with Jenkins, increasing code coverage from under 70\% to 100\%
		\item Developed an Error Provider WPF component using a model-view-viewmodel framework to transmit error messages and display them to the user
	\end{rSubsection} 
	\begin{rSubsection}{Systems Engineering Co-Op at GTRI}{May 2019 - May 2020}{}{}{Atlanta, GA}
		\item Created a LaTeX template for ALQ-213 test procedures, increasing productivity by enabling test procedures to be version controlled
		\item Selected and implemented an acronym library for the test procedure template, allowing test procedure authors to clearly define and tag acronyms
		\item Developed a program to locate and report acronym errors in test procedure documents, eliminating acronym errors
	\end{rSubsection} 
	\begin{rSubsection}{Research Intern at The University of Georgia}{May 2016 - August 2016}{}{}{Athens, GA}
		\item Created augmented reality experiences for user avatar characters to participate in using Unity 3D, demonstrating the future utility of the virtual STEM buddy avatar system
		\item Traveled to the Children’s Museum of Atlanta to demonstrate the virtual STEM buddy project, contributing to the team effort to secure funding
	\end{rSubsection} 
\end{rSection} 
%----------------------------------------------------------------------------------------
%	LEADERSHIP SECTION
%----------------------------------------------------------------------------------------
\begin{rSection}{LEADERSHIP}
	\begin{rSubsection}{Interfraternity Council VP of Coummunications}{November 2020 - Present}{}{}{}
		\item Emphasized committee recruitment for Interfraternity Council technology and public relations sub-committees, increasing committee membership by 900\%
	\end{rSubsection}
	\begin{rSubsection}{Delta Chi Fraternity Recruitment Chairman}{January 2020 - August 2020}{}{}{}
		\item Adapted recruitment strategy to be compatible with the challenges posed by Covid-19 by moving recruitment online using teleconferencing software
		\item Executed virtual recruitment successfully, increasing the number of new members by 106\% from the previous year
	\end{rSubsection}
\end{rSection}
%----------------------------------------------------------------------------------------
%	PROJECTS SECTION
%----------------------------------------------------------------------------------------
\begin{rSection}{PROJECTS}
	\begin{rSubsection}{CMoy Headphone Amplifier}{July 2021 - August 2021}{}{}{}
		\item Selected components and designed a PCB for a portable headphone amplifier based on Chu Moy's research
	\end{rSubsection}
	\begin{rSubsection}{Robot Musician}{May 2021 - August 2021}{}{}{}
		\item Designed software for a robot musician that incorperated elements of DSP and embmedded system design
		\item GitHub: \url{https://github.com/jcline8/3872-project-software}
	\end{rSubsection}
	\begin{rSubsection}{LED Audio Visualizer}{April 2021 - May 2021}{}{}{}
		\item Documented and designed software for an LED audio visualizer using the ARM Mbed
		\item GitHub: \url{https://github.com/jcline8/frequency-vis}
	\end{rSubsection}
\end{rSection}
\end{document}
